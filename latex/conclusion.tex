\chapter{conclusion}

Aline: 

Le datamining m'a rappellé une science inventée par Isaac Asimov dans son livre Fondation. Cette science, nommée psychohistoire, se prétend capable de déterminer les actions futures de l'humanité grâce à des formules mathématiques issues de son histoire passée. Ces théories ne peuvent se prétendre justes, d'après son créateur, que si elles concernent des masses de populations suffisament importantes pour ne pas être parturbées par les comportements individuels. 

De même nous avons pu constater en essayant d'utiliser les méthodes de clustering que la fouille de donnée n'a de sens que si elle concerne un nombre suffisant de données, et qu'avec une fouille efficace il est possible de prévoir, avec un certain poucentage de chance, des comportements.

Un jour, je suis tombée par hasard sur un article parlant d'une famille américaine qui aurait porté plainte contre une chaine de magasin car leur technique de fouille de donnée utilisée pour envoyer de la publicité hyperspécialisée avait ammené les parents à découvrir que leur fille était enceinte. Quelques jours après j'ai reçu une lettre du magasin auquel j'était fidélisée me proposant des réductions sur des produits de consommation courante. J'ai été rassurée de voir qu'aucun des produits proposés ne me correspondait. 
\\

Tout cela m'a incitée à me poser des question sur l'éthique de la fouille de données. Peut-on vraiment tirer toutes les informations des données qui nous sont présentées ou ne faut-il pas, à un moment, savoir laisser de la vie privée dans son étude et ne pas tenter le diable ? Je pense que cette question doit certainement concerner les hommes et les femmes chercheurs en datamining et que, comme dans toute autre science, la fouille de données doit être régulée par des règles d'éthique.



