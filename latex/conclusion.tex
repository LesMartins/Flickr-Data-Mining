\chapter{Conclusion}

\section{Aline}

Le data-mining m'a rappelé une science inventée par Isaac Asimov dans son livre Fondation. Cette science, nommée psycho-histoire, se prétend capable de déterminer les actions futures de l'humanité grâce à des formules mathématiques issues de son histoire passée. Ces théories ne peuvent se prétendre justes, d'après son créateur, que si elles concernent des masses de populations suffisamment importantes pour ne pas être perturbées par les comportements individuels.

De même, nous avons pu constater en essayant d'utiliser les méthodes de clustering que la fouille de données n'a de sens que si elle concerne un nombre suffisant de données, et qu'avec une fouille efficace il est possible de prévoir, avec un certain pourcentage de chance, des comportements.

Un jour, je suis tombée par hasard sur un article parlant d'une famille américaine qui aurait porté plainte contre une chaine de magasin car leur technique de fouille de données utilisée pour envoyer de la publicité hyper-spécialisée avait amené les parents à découvrir que leur fille était enceinte. Quelques jours après j'ai reçu une lettre du magasin auquel j'étais fidélisée me proposant des réductions sur des produits de consommation courante. J'ai été rassurée de voir qu'aucun des produits proposés ne me correspondait.
\\

Tout cela m'a incitée à me poser des question sur l'éthique de la fouille de données. Peut-on vraiment tirer toutes les informations des données qui nous sont présentées ou ne faut-il pas, à un moment, savoir laisser de la vie privée dans son étude et ne pas tenter le diable~? Je pense que cette question doit certainement concerner les hommes et les femmes chercheurs en data-mining et que, comme dans toute autre science, la fouille de données doit être régulée par des règles d'éthique.


\section{Martin}

Je vois le data-mining comme une mission, un challenge, un défi. On dispose d'un nombre incroyable de données. On ne sait pas ce qu'on cherche, mais on sait que tout volume de données contient des secrets à découvrir. On sait que les données qu'on nous a confiées regorgent de trésors à découvrir.

Le plus difficile dans le data-mining, ce n'est pas de tirer des informations des données. C'est de tirer des informations \textbf{utiles} et \textbf{pertinentes}, c'est-à-dire pas seulement des informations qui nous apprennent des choses, mais des informations qui nous apprennent des choses \textbf{utiles}.

Il a été frustrant dans ce projet, de constater que les informations que nous avons tirées ne sont pas forcément utiles~: le fait qu'un gros cluster se trouve près de la place Bellecour ne nous apprend pas grand chose, nous aurions pu deviner avant de regarder les données que cet endroit allait comporter beaucoup de photos. Pour trouver des informations utiles, nous aurions eu besoin de beaucoup plus de temps.

Pour les prochaines années, peut-être serait-il intéressant de proposer aux étudiants de ré-implémenter eux-mêmes les méthodes de clustering. Le défaut de ne pas les implémenter soi-même est que l'on peut très bien appeler les fonctions toutes faites en leur passant des paramètres sans rien comprendre au mécanisme de l'algorithme de clustering utilisé. La ré-implémentation, bien que chronophage, est un moyen sûr de s'approprier les algorithmes et de les comprendre.
