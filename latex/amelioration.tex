\chapter{Résultats, axes d'amélioration et limites}
\section{Interprétation des résultats}

Notre étude est une clusterisation des espaces en fonction de la localisation géographique.
\\

Elle peut déjà permettre à la commune du grand Lyon de repérer quels espaces ont tendance à être des lieux de prise de photos et quels autres ne le sont absolument pas et ainsi choisir d'appuyer les transports sur les espaces les plus pris en photo ou au contraire d'essayer de faire découvrir des endroits intéressants méconnus du public.

\section{Axes d'amélioration}
En alliant les dates avec la localisation, il pourrait être possible de détecter quels lieux auraient tendance à être plus pris en photo à une certaine période de la journée, du mois ou de l'année.
\\

En alliant les utilisateurs avec la localisation, il pourrait être possible de détecter des comportements de photos et ainsi de suggérer à un photographe d'autres endroits où ceux qui ont déjà photographié les mêmes lieux ont également été.
\\

En alliant dates, utilisateur et localisation, il pourrait être possible de déterminer des itinéraires de photos et pourquoi pas d'adapter les transports à ceux-ci.
\\

Les analyses d'image, de légende et de tag pourraient également être utiles pour identifier ce qui est réellement pris en photo (enlever les hors-sujet, séparer les photos de bâtiments juste à côté de vues au loin, etc.).

\section{Limites}
Le plus grand frein au projet a avant tout été le temps. Nous avons disposé de deux séances de 4 heures suivies de deux semaines sans séance pour travailler et rendre le compte-rendu de projet. 
Travaillant en binôme, nous avions donc un temps cumulé en séance de 16~heures à nous deux. Malgré la possibilité de travailler hors séance, nous n'avons pas pu étendre beaucoup le temps passé sur le projet, car d'autres projets se présentaient également à l'ordre du jour.
\\

Le deuxième frein a été les moyens. Pour pouvoir exploiter l'ensemble des données dont nous disposions il aurait fallu pouvoir analyser de manière sémantique les mots employés dans les textes et analyser les informations à partir des photos données. De même les données floues dans les positionnements empêchent de clusteriser de manière optimale, puisque ce manque de précision entraine un fort décalage entre l'échelle des latitudes et des longitudes. Même l'interprétation des différents champs année mois jour heures minutes en type date, bien qu'elle ne soit pas aussi complexe qu'une analyse de texte, demande du temps. Il faut trouver la bonne forme sous laquelle convertir et calculer.
Il n'est pas complexe de faire tout cela, mais le temps nous a manqué.
\\

Le troisième frein a été les données. Bien qu'ayant obtenu 84000 lignes d'information et ayant déjà du mal à toutes les afficher, c'était finalement assez peu pour en extraire de vraies tendances.
\\

Enfin le dernier frein, positif, est que nous avons choisi de faire le sujet avec des outils différents de la plate-forme Knime à laquelle nous avions été formés (bien que nous l'ayons exploitée en début de projet pour visualiser la forme des données). Il a donc fallu intégrer dans notre projet la formation aux nouveaux outils employés (API Google Map, scikit sous python, méthode Meanshift).

